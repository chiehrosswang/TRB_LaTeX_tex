\section{Introduction}
The \trbcite{TRBGuide} has unique and somewhat arbitrary requirements for
papers submitted for review and publication. While the initial submission is
required to be in PDF format, submissions for publication in Transportation
Research Record must be in Microsoft Office format. On top of this, the
manuscripts must be line-numbered, captions are bolded and employ atypical
punctuation, and the references must be numbered when cited and then printed in
order. More details about the manuscript details can be found
online at~\url{http://onlinepubs.trb.org/onlinepubs/AM/InfoForAuthors.pdf}.

It is assumed that the readers of this document have some significant level of
experience in \LaTeX~and \verb1bibtex1. As use of literate programming becomes
more widespread in engineering and planning, it is possible that this template
may need to be made more robust.


\subsection{History}
David Pritchard posted the original versions of this template in 2009 and
updated it in 2011, soon after TRB began allowing PDF submissions. Gregory
Macfarlane made significant adaptations to it in March 2012, allowing for
Sweave integration and automatic word and table counts. Ross Wang automated the
total word count and made some formatting modifications in July 2015. Version
2.1.1 has been made available on GitHub in January, 2016.  Version 3.1 has been made available on Github (\url{https://github.com/chiehrosswang/TRB_LaTeX_rnw}) in June, 2017.  Versions 2.1.1 Lite and 3.1 Lite were made available on GitHub (\url{https://github.com/chiehrosswang/TRB_LaTeX_tex}) in June, 2017 for users who do not need R and Sweave functions provided in the original verions.  


\section{Features}
The template has a number of features that enable quick and painless manuscript
authoring.

\subsection{Title Page}
The standard \LaTeX\ \verb1\maketitle1 command is not very versatile, so we
have replaced it with a \verb1titlepage1 environment. This means that the
writers will be required to manually enter spacings based on the number of
contributors, but the current settings (12pt between authors, 36pt before, and
60pt after them) seems to work well. 

Near the bottom of the title page, TRB requires a count of the manuscript's
words, figures, and tables. This template generates these counts automatically.
The figure and table counts are simply pulled from the \LaTeX\ counters using
the \verb1totcount1 package. The word count feature is not as straight-forward,
as it utilizes a call to the system command \verb1texcount1. Thus to compile
the document writers must enable \verb#\write18# in their \verb1pdflatex1 call.

In the newest version of this template, we added the total count automatically.
The total count basically adds not only the word count, but also the equivalent
count (250 words) for each figure and table.  This is implemented using
a customized command \verb1\totalwordcount1.  Please see the original code for
more information.

\subsection{Page Layout}
The document has 1 inch margins as required, with the author's names in the
left heading and the page number in the right. The authors heading will need to
be edited by the writers; automating this from the title page command is not
currently possible. Paragraphs leading sections and subsections are not
indented, while all subsequent paragraphs in that section are. Section types
are defined as outlined by the \trbcite{TRBGuide}

The document is single-spaced in 12 point Times font. Times New Roman is
a proprietary font and is therefore not available by installation in
open-source software. While the differences between Times variants are
negligible, Times New Roman itself can be used in Mac OSX by compiling under
\verb1xelatex1.

\subsubsection{Line Numbers}
Manuscript line numbering is implemented using the \verb1lineno1 package. There
are options to change the font style and type, but the current settings work
well. Note that the line numbers refresh each page, and that blank lines do not
receive a number. Currently, line numbers and headers are not shown on the
title page, but can be easily added by adding \verb1\pagewiselinenumbers1
command right before the beginning of the title page.

